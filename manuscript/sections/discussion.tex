\section{Discussion}
To understand how open and reproducible data science operates ``under the hood" through ethnographic research and the study of critical triggering dynamics, we have measured the response activity following the impulse of the Astro Hack Week, a one-week hackathon for astronomers. Our results help describe the complex dynamics, behaviors and tradeoffs faced by the actors of this community. We discuss these results in regards to the challenges posed by the development of data driven open and reproducible practices of sciences. 

Our research has characterized a {\it kairos} moment \cite{orlikowski2002s} of open and reproducible science, focused on community joining and integration, with non-trivial spillovers. Our observations suggest that the initial shock, namely, the first day of the Astro Hack Week, or at least the Astro Hack Week itself, gives the impulse for longer term effects.

Similarly, although the exponent $\alpha$, representing the long-memory response (the lower $\alpha$, the slower the decay) seems to be stable and robust across activity types, for the quantitative analyst, it remains unclear whether we observed a genuine feature of the Astro Hack Week, or a more general law of open and reproducible science, or a universal law of open collaboration. The origin of the small exponent ($\alpha \approx 0.25$) requires more investigation. It may stem from the combined effects of cascades of repository creations and critical triggering of contributions. It may also result from complex networks of influence between community members, and types of events \cite{saichev2013hierarchy}. There is little doubt that the nature of events plays a role in the cascading dynamics, yet it remains hardly feasible to investigate them in detail. In addition, there is a multitude of {\it weak} signals, which cannot be captured either because they are systematically not recorded (e.g. talking at a pub, as shown on Figure \ref{fig:pub}), but also from the GitHub data, which might be too entangled or not significant enough to draw solid conclusions.

\begin{figure}[!t]
\centering
\includegraphics[width=0.9\columnwidth]{figures/pub.jpg}
\caption{Happy hours at Astro Hack Week 2014. Photo by Adrian Price-Whelan. (source: \url{http://astrohackweek.github.io/blog/astro-hack-week-wrapup.html})}
\label{fig:pub}
\end{figure}

GitHub, like other social coding platforms, has proven to be a very efficient toolset to enhance open and reproducible science collaboration, and the GitHub team has undertaken efforts for a better integration of collaborative tools for computational science, as reflected also by the high adoption rate of the IPython Notebook in recent years.\footnote{Very recently, GitHub has integrated the IPython Notebook viewer (i.e., NB viewer) in its interface, in order to visually render all notebooks stored in GitHub \cite{notebook_rendering}.}

Our research has shown, one more time, to the importance of material and imaginary structures in the production of science \cite{callon1981unscrewing,shapin1985leviathan}, embedded here in the specific socio-technical interactions of the hackathon. Our results also point to the rapid evolution of the practices and the re-definition of the figure of scientists toward what has sometime been described as {\it Pi} ($\pi$) shaped scientists, i.e., scientists who have enough domain science and data science skills to tackle new problems and who � by their interdisciplinary scope -- redefine the contours of scientific collaboration. Answering questions on a forum about his practice of data science Astrophysicist Bloom noted that he ``had just finished doing a 3-hour machine learning lecture/session for astronomers at an awesome Astro Data Hack week in Univ. Washington", reframed his vision of collaborative work inside the University:

``{\it The idea that in collaborating with methodological scientists, I, as a domain scientist, must bring an important domain problem to the table that also could lead to novel insights/work for methodological scientists. That is, I�m not going to ask a computer science professor to stand up a Spark cluster and manage it just so I can do some cool astro discovery. That�s a mundane, intellectually unexciting endeavor for them even if it�s critical for me. Likewise, I wouldn�t want to spend much time helping a stats professor develop a novel new stats tool on an astro dataset where the question being asked is a trivial one to answer with existing tools.}" \cite{MooreFoundation_DDD2015}

Each hackathon is a unique experience bringing its own kind of people. It remains to be seen how our results could be generalized, and on the contrary, how they constitute the Astro Hack Week footprint. In addition, after this first round of descriptive results, some questions need to be more problematized. As Borgman noted in {\it If Data Sharing is the Answer, What is the Question?} \cite{borgman2015}, in academia practices of open sciences, including data sharing, remain the exception rather than the rule, and pointed the need of critical thinking about the practiced engaged in the open and reproducible science. 

As the 2014 edition of the Astro Hack Week has been considered successful, the 2015 edition is in preparation. During this week, it would be interesting to see if a research protocol could be used to define a more systematic and standardized data collection -- both quantitative and qualitative -- which would allow deepen our understanding of these specific cases of collaborations. To help understand the challenges of open and reproducible science, a more detailed description of the collaboration model that takes into account this emergent practices could be proposed, following e.g., Lee and Paine (see Figure 3 in \cite{lee2015matrix}).

%\begin{figure}[!t]
%\centering
%\includegraphics[width=0.8\columnwidth]{figures/lee2015cscw.png}
%\caption{The Model of Coordinated Action (MoCA) and its seven dimensions with the end points of each continuum (borrowed from Lee \& Paine; authorization to reproduce pending).}
%\label{fig:pub}
%\end{figure}

As the practice of science changes with and on open collaboration platforms, we shall also witness the evolution of activity patterns, when considering the underlying software used by scientists. Further investigation is necessary to delineate how these increasingly online and web-based tools will actually impact the science practice, and whether it is possible to anticipate their evolution.

%Our approach, which combines in-depth modeling of actions taken by participants and their dynamics, before, during and after the AstroWeek, with a typical qualitative gathering of ethnographic evidence, brings a broad context view of the long-memory dynamics observed. We have characterized a {\it kairos} moment \cite{orlikowski2002s} of open and reproducible science, focused on community joining and integration, with non trivial spillovers and long-term effects. Nevertheless, our observations suggest that the initial shock, namely, the first day of the AstroWeek, or at least the AstroWeek itself, gives the impulse. Therefore, at least in theory, the larger the initial impulse the more important the long-term response. There are two ways to enhance this peak activity: Either increase the number of participants, or find ways to increase activity with the same number of participants. In practice, tuning the impulse as a parameter, may be more tricky: More participants increase coordination costs, and it is unclear how to boost activity. The impulse may also be conditioned to previous exposure to  collaboration and social interactions.

%Similarly, although the exponent $\alpha$, representing the long-memory response (the lower $\alpha$, the slower the decay) seems to be stable and robust across activity types, it remains unclear whether we observed a genuine feature of the AstroWeek, of a more general law of open and reproducible science, or a universal law of open collaboration. The origin of the small exponent ($\alpha \approx 0.25$) requires more investigation. It may stem from the combined effects of cascades of repository creations and critical triggering of contributions. It may also result from complex networks of influence between community members, and even types of events \cite{saichev2013hierarchy}. There is little doubt that the nature of events play a role in the cascading dynamics, yet it remains hard to investigate in detail. In addition, there are multitudes of {\it weak} signal, which cannot be captured either because they are systematically not recorded (e.g. talking at a pub, as shown on Figure \ref{fig:pub}), but also from the GitHub data, which might be too entangled or not significant enough to draw solid conclusions.



%This is precisely where the ethnography approach plays a fundamental role to provide context, not only for the dynamics observed and modeled, but also to guide the expert in quantitative social dynamics, in order to ask the good questions and search meaningful information, which might be buried in the ocean of data. For instance, an interviewed senior data scientist (P4), reported that while it was an investment for him to teach at the Astro Hack Week, this event allowed him build social ties with another senior scientist. These social ties have turned into a research project (posted as a repository on GitHub), several months later. In principle, this information could be dug out of the data, but it is less sure whether it would be worth launching a quantitative analysis on these kinds of very long-term follow-up events, without enough contextual information. Having this information in hand, new research paths on the very long-term effects of a hackathon may be quantitatively investigated in the future.
%
%Conversely, the quantitative approach was useful for fact checking of some biased perceptions by interviewees. For instance, a participant believed that activity followed a step function, i.e. little activity before, a lot during, and again little activity after the Astro Hack Week. He was not conscious that follow-up activity really existed in the way we have described.
%
%More broadly, each hackathon is a unique experience bringing its own kind of people. It remains to be seen how the results presented here generalize, and on the contrary, how they constitute the footprint of the Astro Hack Week. Similarly, a larger sample of investigated hackathons would help better identify the rules commons of all hackathons, and at the same time, their unique specificities.
%
%Although GitHub, as well as other social coding platforms, has proven to be very efficient platforms for open and reproducible science collaboration, efforts have been undertaken for a better integration of tools for computational science. For instance, the IPython Notebook has been credited to be ``great for working through things interactively, virtually all [my] work starts here.� Very recently, GitHub has integrated the IPython Notebook viewer (i.e., NB viewer) in its interface, in order to visually render all notebooks stored in GitHub \cite{notebook_rendering}. This example opens the question of the importance of tools in the open and reproducible science process, and how their evolution may additionally change the practice of science. As the practice of science change on open collaboration platform, we shall also witness the evolution of activity patterns, when considering the underlying software used by scientists. Further investigation is necessary to delineate how these increasingly online and web based tools will actually impact the science practice, and whether it is possible to anticipate the evolution of these tools.


%For example, a pattern recognition tool for search and identification of stars, developed by astronomers, may be repurposed for mapping aerial satellite images \cite{kapadia} {\bf [not sure if this example is best]}.

%- find long-term traces of activity (common projects may be initiated months after initial physical meeting $\rightarrow$ kyle interview saying that he has started something with Phill Marshall). Resonates with the very slow (theoretical) decay.


%It's unclear how the social ties established on the long term, triggering collaboration between a subset of the community several months or even years after the Astro Hack Week, may be traced back through the empirical dynamics, to the original encounter. However, the structure interview may greatly help bridge this gap. For instance, when asked about long-term spillovers and benefits of the Astro Hack Week, a senior participant indicated that he started a research project in collaboration with two other senior data scientists, whom he had met and got to know during the Astro Hack Week. Such indication, may help trace meaningful ties between events, which otherwise would be buried into the ambient activity, and for which getting statistical significance of causality, may simply be impossible.


%one point statistics: we shall see it what fashion it will repeat in Astro Hack Week 2015, or in similar events. We shall not presume that all events have the same response dynamics (we have preliminary evidence that another event, involving the astro community has happened earlier in 2014, with slightly different dynamics). We shall on the contrary leverage our proposed method combining quantitive modeling with structured interviews, to bring deep insights, on the complex dynamics and social interactions, occurring during data science events.
