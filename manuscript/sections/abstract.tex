As academic research gets increasingly computational, open and interdisciplinary, studying the practice of science requires to account for heterogenous and self-organized online communities, which leave complex --yet data-rich-- footprints. Considering a community active in astronomy, we studied the dynamics triggered by the Astro Hack Week 2014 (AstroWeek). We found that past the initial impulse, the AstroWeek community triggered long-term follow-up activity. We illustrate how this long-term activity is only tangentially related to the AstroWeek, but rather stems from exposure to a new practice of science, and from the build-up of new social ties. We also document the ``costs"  of senior data scientists spending time teaching data science to lay participants. Our approach, combining state-of-the-art modeling of social dynamics, and ethnography, as a gold-standard approach for STS studies, helps inform how communities build-up, and how open and reproducible science is achieved.