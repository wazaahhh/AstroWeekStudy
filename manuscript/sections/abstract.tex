As academic research gets increasingly computational, distributed, interdisciplinary and open, studying the practice of science requires to account for heterogeneous and self-organized online communities, which leave complex -- yet data-rich -- footprints. Considering a community of scientists committed in open and reproducible science, and active across domains of cosmology, astronomy and astrophysics, we studied the dynamics triggered by a specific community gathering: the Astro Hack Week 2014. We found that past the initial impulse, the Astro Hack Week community triggered long-term follow-up scientific activity, stemming from a socio-technical assemblage of exposure to a new set of collective scientific practices, such as hackathons, growing adoption of collaborative tools and methods across disciplines (e.g., machine learning, statistics, computing). Our approach, combining modeling of social dynamics and ethnography, helps inform how communities build-up and gather around common goals, as well as practices, for a more open and reproducible science.
