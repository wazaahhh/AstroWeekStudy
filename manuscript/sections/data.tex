\section{Data}
\label{sec:data}
Social coding platforms are the backbone infrastructure for storing, sharing, commenting and debugging software. In open and reproducible science, it is similarly the place where code produced for scientific purpose is shared. GitHub is one of the major platforms delivering such service, and it is the one, which has been chosen by the AstroWeek community.

The basic code container is called a {\it repository}. In general, a repository corresponds to logical unit, i.e., a standalone program module designed for a specific purpose. During the AstroWeek, repositories were created for solving specific scientific or computational problems, but also for organizing the AstroWeek: For example, the website and the blog were published from code stored on GitHub repositories. 

When actors, make a contribution to a project, the corresponding repository is updated accordingly and modifications are recorded through {\it events} \cite{github_event_types}. These events include {\it creating} a repository, {\it forking} a repository for the purpose of making own modifications, submitting code (i.e., {\it push}, {\it pull request}), managing {\it issues} and their flow of comments ({\it commit comment, issue comment}). In total, GitHub has 25 types of events, which also include some social features (e.g., {\it follow} a contributor, {\it watch} a repository), which are beyond the scope of this study.

30 out of 39 registered participants used GitHub during the AstroWeek \cite{astroweek_participants}. Only one participant created a Github account during the Monday afternoon breakout session on GitHub and Git (the version control software underlying GitHub), suggesting that most participants had already a previous contact with version control and social coding platforms. It is unclear however whether the 9 other participants, deliberately did not use GitHub or contributed on Hackpad instead \cite{astro_hackpad}. We collected all events triggered by the actions of the 30 participants with a GitHub account, during the Astro Hack Week, but also the records of events triggered by the same participants, during the period of six months before and six months after the AstroWeek (i.e., between March 15th, 2014 and February 20th, 2015). Events, with their timestamp, the actor responsible for the contribution and the associated repository were gathered from the GitHub Archive \cite{github_archive}. We investigated the dynamics of four measures of activity, before, during and after the Astro Hack Week:  

\begin{itemize}
  \item {\bf Repositories} created (count per week): The creation of a repository, means the starting point of a new project, or in case of a {\it fork}, the continuation of an existing project away from the community. A fork may be {\it merged} again later with the main repository, when the changes made in an autonomous way have been accepted by the broader community. A fork can also remain stand-alone, and a new community may aggregate around the forked repository.
  \item {\bf Events} (count per day): They reflect interactions of individuals with the community. Events are usually jointly related to an actor and a repository and may be classified by type (e.g., {\it push}, {\it pull request}, {\it issue}).
  \item {\bf Active Contributors} (count per day): show the level of community mobilization for a given day.
  \item {\bf Active Repositories} (count per day): inform on the span of contributions by active members of the community.
\end{itemize}

These four metrics provide the elementary data to understand and model the dynamics triggered by the AstroWeek, as an exogenous event with short- and long-term effects on the community of astronomy scientists, involved in open and reproducible science. 

%We then study the succession of events for each repository created during, or in relation with the AstroWeek.
%{\bf [insert a table with metrics before, during, just after, and later after the astroweek]}