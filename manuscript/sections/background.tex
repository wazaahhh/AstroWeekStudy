\section{Background}
As computing tools get generalized in larger scientific communities, the lab has progressively evolved with new tools, but also with a refashioned vision. The move towards more open and reproducible practice of science, supported by online collaboration tools, is one of the many artifacts of the fast-paced evolution of science practice. It should not be a surprise that scientific practice has finally reconnected with the open collaboration approach -- the generalized version of open source software development, which was itself deeply inspired by the scientific approach \cite{raymond1999cathedral}. 

Challenges faced by open and reproducible science are, in many ways, similar to those faced by computer supported collaborative work (CSCW). These research and practice challenges include, understanding motives deeply governed by intrinsic motivation \cite{vonKrogh2012}, solving social dilemma in complex settings \cite{baldwin2007}, and overcoming technical and social barriers, which may prevent the growth and the renewal of communities \cite{halfaker2013}. And since open source organizations are rarely formally organized in hierarchical fashions, one of the major challenge for research, is precisely, to decipher the nature of interactions, and their contributions to the complex dynamics of contributions.

Here, we review the CSCW and organizational science literature, which relates most closely to challenges faced by open and reproducible data science. We recall efforts made to investigate and model the complex dynamics and bottom-up properties of open collaboration. We finally tie back to the common research body on CSCW related to open and reproducible science.

\subsection{Open Collaboration Platforms \& Github}
Most today's open collaboration is supported by online platforms, may it be an IRC chat, a mailing list, a forum, a Facebook group, or in its most sophisticated way, a repository on a social coding platform. Some open collaboration groups combine some of these tools, according to their immediate needs (e.g., chats), for issues regarding organization or of broad interest (e.g., mailing lists), or for keeping track of the evolution of their artifact (e.g., document version control). Free and commercial online platforms supporting open collaboration, combine some of these tools. For instance, GitHub integrates document version control, wikis, issue trackers, social features (connect and follow fellow contributors), and Web publishing features (e.g., GitHub Pages).

The impact and the performance of open collaboration platforms, such as GitHub, have been studied from a variety of perspectives including looking at user interactions and evaluation of the contributions \cite{dabbish2012social,tsay2014influence}, as well as how diversity increases group productivity \cite{chen2010}. More broadly, collaboration has been tackled as a knowledge creation practice \cite{bercovitz2011mechanisms}, emplacing variations in the ways people work and collaborate \cite{hayes2011organizational}. The impact of collaboration on knowledge creation has also been addressed\cite{doloreux2012collaboration},as well as the difficulties of working in interdisciplinary and collaborative contexts \cite{bessner2015organizing}, establishing in the way a shared corpus of references \cite{stalzer2015preliminary}.  In particular, socialization in open source software communities seems to play an important role for the success of collaboration  \cite{ducheneaut2005socialization,von2003community}

Although the fine-grained mechanisms driving contribution remain largely unknown, it was found that online collaboration projects exhibit critical cascades of activity \cite{sornette2014much}: In many projects, one contribution will trigger on average another event in the future, hence leading to self-sustained chains of contributions through social interactions \cite{saichev2013hierarchy}. These results are reminiscent of activity dynamics observed in many social networks (e.g., views on Youtube) \cite{crane2008}, as well as for Wikipedia edits following page creation \cite{wilkinson2007}, or following breaking news \cite{keegan2013hotoff}.

\subsection{Community Building, Joining and Renewal}
Yet, community building remains one of the most sensitive problems in open collaboration. Initiating the knowledge sharing process is fragile \cite{gachter2010initiating}, and organization requires constant adaptation as the community grows \cite{shah2006motivation,o2007emergence}, with new members joining with a broad variety of intrinsic and extrinsic motivations \cite{von2003community}, which are still hardly explained. This is furthermore a concern that undertaking work without tangible reward was shown to be a important reason why groupware systems fail \cite{grudin1988cscw,orlikowski1992learning}.

While socialization is often perceived as a pre-requisite for efficient collaboration, individual motives may actually help socialization and collaboration: Situated learning \cite{lave1991situated} was to be an important factor for socialization and sustained participation of newcomers \cite{fang2009understanding}. Finding the right (sub-)community to leverage best situated learning and {\it collaborativeness} has been recently studied on Wikipedia \cite{klein2015virtuous}, where on-boarding new contributors has been a recurrent problem \cite{halfaker2013}. Generally, on-boarding open collaboration communities is challenge, which requires learning a number of un-written rules and social norms, which has even led to dedicated classes at the university, such as the one proposed at UC Berkeley in 2013, on open collaboration and peer-production \cite{i290_ocpp}. 

\subsection{Hackathons for Open and Reproducible Science}
Upon transitioning from a ``traditional" practice of science to a computational and data driven open and reproducible science \cite{calvert2013collaboration}, with its new codes, social norms, new practices, academia faces similar problems, regarding building, organizing and joining communities.

Misaligned incentives represent an additional barrier to overcome: Beside financial rewards (e.g., wages) open source software developers mainly derive utility through recognition and reputation \cite{shah2006motivation}. For instance, on GitHub, the number of {\it followers} is a symbol of social status \cite{dabbish2012social}. In contrast, scientists who write software are rewarded only indirectly through publications \cite{howison2011scientific,Howison2013incentives}. The problem of additional effort for producing open code is reminiscent of data sharing \cite{trainer2015personal} and may be bound by additional conceptual limits \cite{huang2013meanings}.

Building a community with social norms and reputation systems directly aligned with the effort (i.e., with a reward for the code produced) is therefore a crucial element for the success of open and reproducible science \cite{segal2009software}. Prior studies suggest that community code engagement, through short-term, intensive, software development events, may be an effective way to build sustainable communities and scientific software \cite{trainer2013big}. Such hackathons for open and reproducible science, have already been studied mainly with qualitative methods \cite{trainer2014community}.\\

%From Personal Tool to Community Resource: What�s the Extra Work and Who Will Do It? \cite{trainer2015personal} $\rightarrow$ need for tools for sharing $\rightarrow$ ipython notebook

%The big effects of short-term efforts: A catalyst for community engagement in scientific software \cite{trainer2013big}

%As researchers have noted, collaboration between scientists has been looked at through the �modeling the emergence of collective phenomena from individual interactions, particularly through agent-based models� which �partake of the same conceptual approach in which individuals are taken as discrete and interchangeable 'social atoms' (�) out of which social structures emerge as macroscopic characteristics (viscosity, solidity...) emerge from atomic interactions in statistical physics� (Tommaso and al, 2015)

%Open source projects and their online communication platforms coupled with the code repository serve a similar social network role yet at much smaller scales \cite{madey2002open,crowston2005social}.
















