\section{Background}
As computing tools get generalized in larger scientific communities, scientific work has been evolved with new tools that, in turn, have transformed the way scientists think about their science \cite{leonelli2012introduction,aronova2010big}. The move towards more open and reproducible practice of science \cite{donoho2010invitation,wilson2014best}, supported by online collaboration tools \cite{aragon2011collaborative}, is one of the on ramping artifacts of the fast-paced evolution these science practices. It reconnects with the open collaboration approach -- the generalized version of open source software development -- which was itself deeply inspired by the scientific approach \cite{raymond1999cathedral}. 

In recent years, researchers have recognized that ``{\it science is built upon foundations of theory and experiment validated and improved through open, transparent communication. With the increasingly central role of computation in scientific discovery this means communicating all details of the computations needed for others to replicate the experiment, i.e., making available to others the associated data and code. The ``reproducible research" movement recognizes that traditional scientific research and publication practices now fall short of this ideal, and encourages all those involved in the production of computational science -- scientists who use computational methods and the institutions that employ them, journals and dissemination mechanisms, and funding agencies -- to facilitate and practice really reproducible research}" \cite{donoho2010invitation}.

In Astrophysics, for instance, recent breakthroughs have been made by large interdisciplinary collaborations of geographically distributed scientists, working with complex, and often messy, data \cite{poon2008context}. In a world where the volume, and the multi dimensionality of data has dramatically increased in the last decades \cite{sands2014we}, new tools needed to be invented to allow progress of the scientific endeavor \cite{bercovitz2011mechanisms}. During these complex distributed processes, scientists are redefining the always changing contours of the scientific community, and at the same time, they are transforming the ethos of their profession \cite{mialet2012hawking}.

Challenges faced by open and reproducible data driven science are, in many ways, similar to those of  computer supported collaborative work (CSCW), combined with the necessity to develop scientifically meaningful projects that articulate innovative methods with relevant scientific questions. As Astrophysicist Bloom noted : {\it Any scientist would be proud to say that he or she made just a few truly novel contributions in their lifetime�ones that bear fruit, hold up against scrutiny, and are enduringly remembered. I am hopeful that as we embark on our new data science efforts we will find ways to enable the creation of novel domain science alongside (and enabled by) novel methodological approaches. In turn, I hope we discover data and questions about data that require novel methodological approaches} \cite{bloom2013novelty}. The challenges of open and reproducible data driven science include many aspects of collaborative work previously studied by CSCW researchers, like understanding motives deeply governed by intrinsic motivation \cite{vonKrogh2012}, solving social dilemma in complex settings \cite{baldwin2007}, and overcoming technical and social barriers, which may prevent the growth and the renewal of communities \cite{halfaker2013}. They challenge our capacity to decipher the complex nature of interactions, and their role in the dynamics of contributions.

In the following section, we review the CSCW and organizational science literature, which -- we argue -- relates most closely to the socio-technical challenges faced by open and reproducible data science. We recall efforts made to investigate and model the complex dynamics and bottom-up properties of open collaboration. We finally tie back to the common research body on CSCW related to open and reproducible science.

%As computing tools get generalized in larger scientific communities, the lab has progressively evolved with new tools, but also with a refashioned vision. The move towards more open and reproducible practice of science, supported by online collaboration tools, is one of the many artifacts of the fast-paced evolution of science practice. It should not be a surprise that scientific practice has finally reconnected with the open collaboration approach -- the generalized version of open source software development, which was itself deeply inspired by the scientific approach \cite{raymond1999cathedral}. 

%Challenges faced by open and reproducible science are, in many ways, similar to those faced by computer supported collaborative work (CSCW). These research and practice challenges include, understanding motives deeply governed by intrinsic motivation \cite{vonKrogh2012}, solving social dilemma in complex settings \cite{baldwin2007}, and overcoming technical and social barriers, which may prevent the growth and the renewal of communities \cite{halfaker2013}. And since open source organizations are rarely formally organized in hierarchical fashions, one of the major challenge for research, is precisely, to decipher the nature of interactions, and their contributions to the complex dynamics of contributions.

%Here, we review the CSCW and organizational science literature, which relates most closely to challenges faced by open and reproducible data science. We recall efforts made to investigate and model the complex dynamics and bottom-up properties of open collaboration. We finally tie back to the common research body on CSCW related to open and reproducible science.


\subsection{Open Collaboration Platforms \& GitHub}
Most today's open collaboration is supported by online platforms, may it be an IRC chat, a mailing list, a forum, a Facebook group, or in its most sophisticated way, a repository on a social coding platform. Some open collaboration groups combine some of these tools, according to their immediate needs (e.g., chats), for issues regarding organization or of broader interest (e.g., mailing lists), or for keeping track of the evolution of their artifact (e.g., document version control). Free and commercial online platforms supporting open collaboration, combine some of these tools. For instance, GitHub mainly integrates document version control, wikis, issue trackers, social features (connect and follow fellow contributors), and Web publishing features (e.g., GitHub Pages).

In the field of complex system dynamics, although the fine-grained mechanisms driving contribution have been difficult to capture by quantitative methods, it was found that online collaboration projects exhibit critical cascades of activity \cite{sornette2014much}: In many projects, one contribution will trigger on average another event in the future, hence leading to self-sustained chains of contributions through social interactions \cite{saichev2013hierarchy}. These results are reminiscent of activity dynamics observed in many social networks (e.g., views on YouTube) \cite{crane2008}, as well as for Wikipedia edits following page creation \cite{wilkinson2007}, or following breaking news \cite{keegan2013hotoff}. 

Meanwhile, social scientists have largely advanced the understanding of complex socio-technical processes that proceed to the co-creation of collaborative environment, scientific or not \cite{anandarajan2010research,doloreux2012collaboration,goggins2014designing,ribes2013artifacts,travlouethnographies}. For instance, the impact and the performance of open collaborative platforms, such as GitHub, have been studied from a variety of perspectives including looking at user interactions and evaluation of the contributions \cite{dabbish2012social,tsay2014influence}, as well as how diversity increases group productivity \cite{chen2010}. More broadly, collaboration has been tackled as a knowledge creation practice \cite{bercovitz2011mechanisms}, emplacing variations in the ways people work and collaborate \cite{hayes2011organizational}. The impact of collaboration on knowledge creation has also been addressed \cite{doloreux2012collaboration}, as well as the difficulties of working in interdisciplinary and collaborative contexts \cite{bessner2015organizing}, establishing in the way a shared corpus of references \cite{stalzer2015preliminary}. In particular, socialization in open source software communities seems to play an important role for the success of collaboration \cite{ducheneaut2005socialization,von2003community}.

\subsection{Community Building, Joining and Renewal}
Community building, is yet one of the most sensitive problems in open collaboration. Initiating the knowledge sharing process is fragile \cite{gachter2010initiating}, and organization requires constant adaptation as the community grows \cite{shah2006motivation,o2007emergence}. New members join with a broad variety of intrinsic and extrinsic motivations \cite{von2003community}, which are still hardly explained. This is furthermore a concern that undertaking work without tangible reward was shown to be an important reason why groupware systems fail \cite{grudin1988cscw,orlikowski1992learning}.

While socialization is often perceived as a pre-requisite for efficient collaboration, individual motives may actually help socialization and collaboration: Situated learning \cite{lave1991situated} was found to be an important factor for socialization and sustained participation of newcomers \cite{fang2009understanding}. Finding the right (sub-) community to leverage best-situated learning and capacity to collaborate has been recently studied on Wikipedia \cite{klein2015virtuous}, where on-boarding new contributors has been a recurrent problem \cite{halfaker2013}. Generally, on-boarding open collaboration communities is challenge, which requires learning a number of unwritten rules and social norms, which has even led to dedicated classes at the university, such as the one proposed at UC Berkeley in 2013, on open collaboration and peer-production \cite{i290_ocpp}.

\subsection{Hackathons for Open and Reproducible Science}
Upon transitioning to a computational data-driven open and reproducible science \cite{calvert2013collaboration}, with its new codes, social norms, new practices, academia faces similar problems, regarding building, organizing and joining communities. Misaligned incentives represent an additional barrier to overcome: Beside financial rewards (e.g., wages), open source software developers mainly derive utility through recognition and reputation \cite{shah2006motivation}. For instance, scientists who write software are rewarded only indirectly through publications \cite{howison2011scientific,Howison2013incentives}. The problem of additional effort for producing open code is reminiscent of data sharing \cite{trainer2015personal} and may be bound to additional conceptual limits \cite{huang2013meanings}.

Building a community with social norms and reputation systems directly aligned with the effort (i.e., with a reward for the code produced) is therefore a crucial element for the success of open and reproducible science \cite{segal2009software}. Praised or criticized, hackathons have become one of the main setting of collective co-production in recent years \cite{guillaud2012}. Prior studies suggest that community code engagement, through short-term, intensive, software development events, may be an effective way to build sustainable communities and scientific software \cite{trainer2013big}; While others have pointed that the ``failure of hackathons to deliver viable, long-lasting products is a valid complaint" \cite{wood2013}. Such events for open and reproducible science, have been studied with both qualitative and qualitative methods \cite{trainer2014community,fiore2014,irani2015hackathons}.\\

With this framing in mind, the Astro Hack Week 2014 seems to be a good field research that encapsulates some new trends of these large scientific collaborations. It represents a special moment of community engagement, combining socialization \cite{fang2009understanding}, and joining associated with situated learning \cite{lave1991situated}. As one of the participant of the Astro Hack Week recalled: ``{\it hacking can mean deconstructing something that already exists, ``hacking" into it and modifying it to fit your purpose. Or it can mean building a tool as quickly as possible, not fussing over details, just getting something working, exploring and experimenting as you go"} \cite{angus2014}. In this particular context, the Astro Hack Week participants did mostly the latter, and the program was specifically designed to provide junior scientists with important tools to interact within this large interdisciplinary collaborative environment: Mornings where dedicated to lectures (data science, statistics or machine learning) and afternoon to building/hacking projects sessions. This notion of special moments resonates with productive bursts pervasively found in open source software projects \cite{sornette2014much}. 

The Astro Hack Week may be considered as a {\it kairos} moment, by opposition to routine {\it chronos} of the everyday scientific practice, or of time consumption \cite{orlikowski2002s}.This {\it kairos} event was precisely designed to help astronomers pause their routine work, in order to learn and practice tools useful for open and reproducible science (mainly IPython Notebook for collaboration, and GitHub for broad sharing).  The goal was also to further build a community of data scientists who inhabit the same space, share the same language(s), operate within a common social, technical and cultural setting, and � last but not least, have fun being together. As Fiore-Silfvast noted during her field research:
``{\it Astro hackers live in a very dusty, dirty, and noisy environment! Very hard to keep clean and elaborate measures are taken to obtain a signal. But when the signal is too strong or the data too clean, there is a feeling of mistrust. The common language is Python, although there are many other dialects, some entirely made of acronyms, others sound like common names, such as George and Julia. When talking there is always some form of data, documentation or model that mediates the conversation, whether it is on the white board, on the screen, or through representational gestures. Although most people are studying something that has to do with astronomy, they can literally be operating on ``different wavelengths"! Astro hackers play with ``toys" and ``fake data" as much as ``real world data"! Coffee and beer fuel interactivity!}" \cite{fiore2014} 

For participants, Astro Hack Week can been be seen as a {\it shock} or {\it impulse}, with quantitative and qualitative implications, through the dynamics of contributions resulting from this event. Thirty-nine scientists, i.e., astronomers, cosmologists, physicists and data scientists, of various seniority in their field, from junior astrophysics students to senior programmers met for 5 days, devised, designed and coded projects, with most tangible events recorded. These events include creating a repository for a project, pushing a new version of a project with updated source code, commenting and pointing some issues. They were recorded on GitHub \cite{github_event_types} and may be parsimoniously traced. While their detailed filiation --what event has triggered one or several follow-up events-- remains most often ambiguous, the aggregate event dynamics inform on the short- and long-term quantitative consequences of the Astro Hack Week. In particular, they shed light on the nature of triggering events, which are artifacts of individual and collective actions.


%From Personal Tool to Community Resource: What�s the Extra Work and Who Will Do It? \cite{trainer2015personal} $\rightarrow$ need for tools for sharing $\rightarrow$ ipython notebook

%The big effects of short-term efforts: A catalyst for community engagement in scientific software \cite{trainer2013big}

%As researchers have noted, collaboration between scientists has been looked at through the �modeling the emergence of collective phenomena from individual interactions, particularly through agent-based models� which �partake of the same conceptual approach in which individuals are taken as discrete and interchangeable 'social atoms' (�) out of which social structures emerge as macroscopic characteristics (viscosity, solidity...) emerge from atomic interactions in statistical physics� (Tommaso and al, 2015)

%Open source projects and their online communication platforms coupled with the code repository serve a similar social network role yet at much smaller scales \cite{madey2002open,crowston2005social}.
















