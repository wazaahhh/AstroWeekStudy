\section{Introduction}
In recent years, forms of collaborations and cooperation in science have been transformed. As computing tools get generalized in larger scientific communities, the lab, which used to be studied in the classic Science Technology and Society (STS) literature \cite{latour1987science,houdart2008cour} has been progressively replaced by new tools and new settings, such as a refashioned vision \cite{galison1996disunity,kitchin2014big} about the way science should be practiced \cite{calvert2013collaboration,leonelli2012introduction}. Additionally, while the last decades have been dominated by closed-source scientific software (e.g., MatLab, Mathematica), number of tools and libraries for research nowadays build upon community maintained open source programs and libraries, such as R, Numpy, Scipy, IPython Notebook, or Jupyter. These libraries are also increasingly publicly shared, for peer-review, for thorough reproduction of research results, and for reuse in other contexts, and by other researchers. The lines between scientific software and libraries used for research are increasingly blurred, into the same complex ecosystem of code sharing. Robust scientific tools are increasingly achieved through the integration of heterogenous hacks and disruptive ways of tackling problems. In this world of open and reproducible science, a researcher may be at the same time a contributor, a debugger, a user, and a promoter of a scientific library, in addition to conceiving research protocols, gathering data, testing hypotheses, and presenting results in scientific articles.

The way open and reproducible science evolves is a challenge, if not a paradigm change, for STS studies. In recent years, however, the development of collaborative tools for scientific research has generated a large amount of data that may help bridging the gap between qualitative and quantitative methods. As Tommaso et al. \cite{tommaso2015} {\bf [missing reference for Tommaso]} recalled, both the phenomenon of traceability and the capacity to re contextualized the data that have been collected allow the emergence of a new form of ethnography, which is both qualitatively and quantitatively informed. Such method seems particularly relevant to study the complex dynamics of scientific collaboration that social coding platforms, such as GitHub, are making possible and visible in the context of computational intensive scientific projects. 

While open collaboration, as well as open and reproducible science, seems to be a genuine means and feature witnessed broadly in highly connected societies  \cite{benkler2011penguin}, understanding the complex dynamics shaping self-organized communities has remained a challenge, despite the abundance of publicly available data available. Most often, large scale data mining (quantitative) studies lack sufficient context to extract the meaning of actions performed, while (qualitative) studies focused on extracting context are too partial and focused, thus preventing from getting a better big picture without losing context. Here, we address this methodological gap, by combining state-of-the-art modeling of social dynamics, and ethnography, as a gold-standard approach for STS studies, in order to inform how communities build-up, and how open and reproducible science is achieved.

Like for most instances of open collaboration, academia needs to build a strong and large community of scientists. Important efforts have been undertaken across fields by universities (e.g., Software Carpentry \cite{software_carpentry}, or by discipline (across institutions) \cite{calvert2013collaboration}. In this paper, we focus on the Astro Hack Week (AstroWeek), which was held at the University of Washington, from September 15-19, 2014, with the support the Moore Foundation, the Sloan Foundation, and the UW eScience Institute. A large quantity of events that punctuated the AstroWeek, have been recorded on GitHub \cite{github}, a social coding platform, which rapidly transforms the way in which scientists are collaborating \cite{gerson2013integration}. GitHub is promoting new way to collaborate in data driven scientific environment where science is being redefined with some value of openness and reproducibility \cite{ducheneaut2005socialization} -- which include code accessibility  and verification \cite{hayden2015rule,proebsting2015repeatability}.

Looking at the dynamics of contributions by actors, we show that past the initial impulse, AstroWeek triggered long-term follow-up activity. We illustrate how this long-term activity is only qualitatively tangentially related to the AstroWeek, and rather stems from the exposure to a new practice of science, and from the build-up of new social ties. We also document the ``costs"  of senior data scientists spending time teaching data science to lay participants.

The rest of this paper is organized as follows. We first further expose the reader to the STS literature in relation with the emergence of platforms and tools, tuned for a rather self-organized practice of science, as well as research on the complex dynamics of computer supported collaborative work (CSCW). We then present a method, which combines in-depth modeling of actions taken by participants, before, during and after the AstroWeek, with a typical qualitative gathering of ethnographic evidence, which brings a broader context view of the long-memory dynamics observed. Data employed and results are then presented, and discussed. We finally conclude with limitations and future research directions.

%During the days the time was divided between the sessions dedicated of teaching and learning of coding, statistics, and skills required for data analysis of large data set and the afternoon where dedicated to specific projects and breakout sessions. 


%To address the problem of community building and onboarding, we study the community of astronomers {\bf [astrophysicists? cosmologists?]}, learning and practicing data science. In particular, we focus on the Astro Hack Week, which was held at the University of Washington on September 15-19, 2014 (with the support the Moore Foundation, the Sloan Foundation, and the UW eScience Institute). 


%Here, we shall document how a one-week hackathon (the Astro Hack Week), organized for the purpose of engaging scientists in the practice of open and reproducible data science, has short-, mid-term and long-term spillovers {\bf [not sure for the long-term]}, as a result of critical cascades of repository creations and forks \cite{}, as well as contributions to broader data science community {\bf [it's not clear here, but the goal is to convey the idea that while projects during the Astro Hack Week, may be short lived, the mere fact of having people interact (physically), creates opportunities for future stand-alone contributions or even collaborations (c.f., testimony of Kyle)]}

