\section{Conclusion}
With the development of open and reproducible science, new research practices have emerged. These practices rely on writing software and libraries for the purpose of solving scientific problems, often associated with massive amount of data. To help researchers familiarize with this {\it new way} of doing science, hackathons are increasingly organized. Here, we studied the dynamics triggered by the Astro Hack Week 2014. We found that past the initial impulse, involving a trade-off for senior data scientists, the community triggered mid- and long-term follow-up activity, with long-term activity being more related to the acquisition of this new science practice. Our research approach combined quantitative modeling of social dynamics, and gathering of ethnographic materials. This approach has proven cross-nurturing, and may further help enrich research in Science, Technology and Society (STS), focused on the emergence of collaborative approaches to the practice of science.