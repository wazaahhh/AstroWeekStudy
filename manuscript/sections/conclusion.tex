\section{Conclusion}
With the development of open and reproducible science, new research practices have emerged. These practices rely on writing software and libraries for the purpose of solving scientific problems, often associated with massive amount of data. To help researchers familiarize with this {\it new way} of doing science, hackathons are increasingly organized. Here, we studied the dynamics triggered by the Astro Hack Week 2014. We found that past the initial impulse, involving a trade-off for senior data scientists, the community triggered mid- and long-term follow-up activity, with long-term activity being more related to the acquisition of this new science practice. 

In our respective academic traditions, social science and complexity science, information could be dug out of our different data collection processes. On the one hand, for the quantitative analysts, it is questionable whether it would be worth launching a quantitative analysis of this kind with very long-term follow-up events, without gathering enough contextual information. On the other hand, for the qualitative researcher, quantitative theories and modeling provide predictive models of activity and social interactions, which in turn can be observed in field research. In the future we hope to integrate more critical thinking: The ethnographic approach plays a fundamental role to provide context and critical analysis of the dynamics observed. At the same time, the quantitative approach helps frame qualitative questions and target meaningful ethnographic information retrieval. Our approach, combining quantitative and qualitative methods, may thus further help enrich research in Science, Technology and Society (STS), and in complex system dynamics, both focused on the emergence of collaborative approaches to the practice of science.



%With the development of open and reproducible science, new research practices have emerged. These practices rely on writing software and libraries for the purpose of solving scientific problems, often associated with massive amount of data. To help researchers familiarize with this {\it new way} of doing science, hackathons are increasingly organized. Here, we studied the dynamics triggered by the Astro Hack Week 2014. We found that past the initial impulse, involving a trade-off for senior data scientists, the community triggered mid- and long-term follow-up activity, with long-term activity being more related to the acquisition of this new science practice. Our research approach combined quantitative modeling of social dynamics, and gathering of ethnographic materials. This approach has proven cross-nurturing, and may further help enrich research in Science, Technology and Society (STS), focused on the emergence of collaborative approaches to the practice of science.