\section{Method}
Hackathons for open and reproducible science represent these special moments of community engagement, combining socialization \cite{fang2009understanding}, and joining associated with situated learning \cite{lave1991situated}. They may be considered as {\it kairos} moments by opposition to routine {\it chronos} consumption of time \cite{orlikowski2002s}. This notion of special moments resonates with productive bursts pervasively found in open source software projects \cite{sornette2014much}.

The Astro Hack Week (AstroWeek) may be considered as a {\it kairos} moment, which aim was precisely to help astronomers pause their routine work, to learn and practice tools useful for open and reproducible science --mainly IPython Notebook / Jupyter, for collaboration and GitHub for broad sharing-- and to further build a community of data scientists.

The AstroWeek may also be seen as a {\it shock}, with quantitive and qualitative implications, through the dynamics of contributions resulting from this event. 30 scientists met for 5 days, devised, designed and coded projects, with most tangible events recorded. These events include creating a repository for a project, pushing a new version of a project with updated source code, commenting and pointing some issues. They were recorded on GitHub \cite{github_event_types} and may be parsimoniously traced. While their detailed filiation --what event has triggered one or several follow-up events-- remains most often ambiguous, the aggregate event dynamics inform on the short- and long-term quantitative consequences of the AstroWeek. In particular, they shed light on the nature of triggering events, which are artifacts of individual and collective actions.

To investigate the AstroWeek, we use self-excited Hawkes conditional Poisson process \cite{hawkes1974acluster} as a quantitative framework, which typically captures well a variety of social dynamics involving complex human interactions such as online viral meme propagation \cite{crane2008}, open source software development dynamics\cite{saichev2013hierarchy,sornette2014much}, gangs and crime in large American cities \cite{mohler2011}, cyber crime \cite{baldwin2012} and financial contagion \cite{ait-sahalia2010,filiminov2012}. We complement our quantitative findings with a qualitative approach, based on interviews and available material of ethnographic relevance (e.g., pictures, blog posts, tweets).

Self-excited Hawkes conditional Poisson processes \cite{hawkes1974acluster}, are particularly handy to account for exogenous shocks, perturbing a system, and triggering endogenous reactions by this system \cite{crane2008}. The Hawkes conditional Poisson process is defined by the intensity $I(t)$ of events at time $t$, given by
\begin{equation}
I(t)= \lambda(t) + \sum_{i | t_t<t}  f_i \phi(t-t_i)~,
\label{eq:Hawkes}
\end{equation}
where $\{t_i, i=1, 2, ...\}$ are the timestamps of past events, $\lambda(t)$ is the exogenous rate of events, $f_{i}$ is the average fertility of events $i$ that quantifies the number of daughter (first generation) events, and $ \phi(t-t_i)$ is the memory kernel, which reflects the long-memory effects of task prioritization, and economy of time as a non storable resource \cite{maillart2011}. Here, the main exogenous shock is $\lambda(t_{0})$, with $t_{0}$ the first day of the AstroWeek. The second part of the sum in equation (\ref{eq:Hawkes}) represents the endogenous response to exogenous shocks (i.e., here, the response to the initial shock at $t_{0}$). We study the cascading process, expressed in terms of number of events triggered over time. Typical cascades of events, include debugging: A {\it push} event may be imperfect, and shall be corrected by one or several other push events (by the same person or by another community member). Similarly, a {\it pull request} event may trigger some discussion, follow-up modifications, prior to integration in the main body of work (i.e., {\it merge} event).

The Hawkes model is the simplest conditional Poisson process that combines both exogenous shocks and endogenous response. The class of Hawkes models can be mapped onto the general class of branching processes \cite{daley2007}. The statistical average fertility $\langle f_i \rangle$ defines the branching ratio $n$, which is the key parameter: In the sub-critical regime ($n<1$), the average activity tends to die out exponentially fast and the exogenous source term $\lambda(t)$ controls the overall dynamics. At criticality ($n=1$), one commit is on average triggered in direct lineage by a previous commit, corresponding
to a marginal sustainability of the process with infinitesimal exogenous inputs. The super-critical regime ($n>1$) is characterized by an explosive activity that can occur with finite probability \cite{helmstetter2002subcritical,helmstetter2003}. Many social systems have been found to operate close or at criticality, including Youtube video views \cite{crane2008}, book sales on Amazon \cite{sornette2004,deschatres2005dynamics}, and trade dynamics on financial markets \cite{filiminov2012}. One can show that the response dynamics, following an exogenous shock follows a power law decay, 

\begin{equation}
Q(t-t_{0}) \sim 1/(t-t_{0})^{\alpha}~,~t> t_{0}
\label{eq:critical_decay}
\end{equation}

with $Q(t-t_{0})$ the activity after the exogenous shock having occurred at $t_{0}$, and $\alpha$ the exponent. When the triggering regime is sub-critical $\langle f_i \rangle = n < 1$, the response following a shock is delayed by priority queueing, and the time response follows a  power law with exponent $ \alpha \approx 1.5$ \cite{maillart2011}. On the contrary, a slow power law decay ($\alpha < 1$), is a signature of critical triggering, and thus, informs on the nature of individual and collective contribution dynamics \cite{sornette2014much}. In some cases involving social interactions, it has been found that the time-response exponent may be arbitrary small \cite{saichev2013hierarchy}.

With the Hawkes Poisson process framework, we are tooled to understand the dynamics triggered by the AstroWeek, which brought together scientists (i.e., astronomers, cosmologists, physicists and data scientists), of various seniority in their field, from junior astrophysics students to senior programmers. While some people may have known each others beforehand, most had never collaborated together. And since the purpose was to learn collaborative coding, a number of participants had no or limited skills in open and reproducible data science. They participated for the purpose of learning and joining a community to help their practice of scientific research. The AstroWeek schedule was divided between morning sessions dedicated to teaching and learning of coding, statistics, and skills required for data analysis of large datasets.Afternoon sessions were dedicated to specific projects and breakout sessions. Because it was a unique event of its own kind, the AstroWeek may legitimately be seen as an exogenous event, bringing new people and building new social and professional ties for the purpose of consolidating the community.
 %an unstructured conference, which was held at the University of Washington from September 15-19, 2014,
%More precisely, the mere goal of the AstroWeek was to interact from flesh, in order to initiate online collaboration. One of the first breakout sessions on Monday, was dedicated to learning the versioning system Git and to interact with GitHub, the social coding platform. GitHub is structured around a set of events, such as {\it push, pull request, issue, and comment} events (see \cite{github_event_types} for a complete list). Thus, measuring event activity on GitHub informs on the outcomes of the short-term whereabouts and the long-term effects of the Astro Hack Week. 

From GitHub data  (c.f., next section), we measure the quantitative changes, which have occurred from the beginning of the AstroWeek, and onwards, first by considering only contributors and Github repositories modified during the week, then by considering all repositories modified by the AstroWeek participants. We consider (i) the immediate contributions dynamics during the AstroWeek, (ii) the long-term spillovers (i.e., after the AstroWeek), but also, (iii) the tradeoffs for senior contributors (with high seniority on GitHub) when they dedicate their time to help on-boarding new data scientists.

%The empirical evidence of dynamics uncovered from the Astro Hack Week, and their aftereffects, have shed light on a number of questions, which may be best addressed through the collection on ethnographical material (e.g., from records on the Astro Hack Week website, blogposts, and tweets), as well as the interview of a senior researcher in astronomy, also deeply involved in the development of tools that promote open and reproducible science.

%Our method combines cutting-edge quantitative modeling of contribution dynamics and state-of-the-art qualitative research, the gold-standard of research in Science, Technology and Society (STS) studies. This cross-nurturing allows some limitations of both methods, in the case of open and reproducible science, involving rather large communities: On the one hand, while the quantitative approach is good for interpreting the effects of a large amount of events, triggered by a variety of contributors, with heterogenous motives, it does a poor job bringing enough context to understand the social implications of these uncovered dynamics. On the other hand, the ethnographic view in STS studies, mostly focused on the qualitative description of detailed succession of actions, and their meaning, do not scale up for the systematic characterization of tens of scientists, contributing to tens of projects, in rather heterogenous and asynchronous ways, which is the increasingly the norm, as the practice of science has become increasingly inter-disciplinary, computational, and based on pooling large amounts of open data.

%Two Main Components
%\begin{itemize}
%  \item Github
%  \item Ipython Notebook
%\end{itemize}
%
%Main dimensions:
%
%\begin{itemize}
%  \item Adoption
%  \item Boost in use
%\end{itemize}

%{\it In this paper we will focus on two tools, or platforms (Gerson, 2012) that seems to rapidly transform the way in which scientists are collaborating: the IPython Notebook (NB) and GitHub. I recent year, the IPython NB has been credited to be �great for working through things interactively, virtually all my work starts here.� To do so we have focused our analysis on a specific event: 


